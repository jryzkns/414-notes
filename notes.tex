\documentclass{article}
\usepackage[utf8]{inputenc}
\usepackage[margin=0.75in]{geometry} % lots more margin
\pagenumbering{gobble} % ignore page numbers

\usepackage{titling}
\setlength{\droptitle}{-0.75in}

\title{CMPT 414 review notes}
\author{}
\date{}

\setlength{\parindent}{0cm}

\usepackage{enumitem}
\usepackage{graphicx}
\usepackage{amsmath}
\usepackage{amsfonts}
\usepackage{hyperref} % for nice looking urls
\usepackage{booktabs} % for making tables
\usepackage{amssymb}
\usepackage{listings}
\usepackage{graphicx}
\usepackage{caption}
\usepackage{subfigure}
\usepackage{multicol}

\begin{document}

\maketitle

\begin{multicols}{2}

\section{Image Processing}
\subsection{Histogram Transformations}
\subsubsection{Linear Stretching}
\subsubsection{Histogram Equalization}
\subsection{Convolution and Filtering}
% overview of frequency analysis
% low pass vs high pass filtering
% mask design
% Applying 2D Mask vs 1D Mask
% linear and shift invariance
\subsection{Smoothing}
\subsubsection{Unweighted Averaging}
\subsubsection{K-nearest Neighbor Averaging}
\subsubsection{Median Filtering}
\subsubsection{Gaussian Filtering}
\subsection{Sharpening}
\subsubsection{Laplacian-based Methods}
\subsubsection{Unsharp Masking}
\subsection{Image Moments}

\section{Template Matching}
% cross correlation vs convolution, how it doesn't matter in cv

\section{Edge Detection}
\subsection{Gradient-Based Methods}
\subsubsection{Gradient Thresholding}
\subsubsection{Roberts Operator}
\subsubsection{Prewitt and Sobel Operators}
\subsection{Laplacian-Based Methods}
% log vs dog
\subsection{Marr's Edge Operator}
\subsection{Canny Edge Operator}
% criterion for good edge detection
% 5 stages of Canny Edge Operator
% \begin{enumerate}
%         \item Smoothing
%         \item Gradient Operator
%         \item Non-maximum Suppresion
%         \item Double Thresholding
%         \item Hysteresis Tracking
% \end{enumerate}

\section{Regions and Segmentation}

\section{Texture Analysis}
\subsection{Statistical Methods}
\subsubsection{Spatial Gray Level Dependence Method}
\subsubsection{Gray Level Run Length Method}
% 5 texture measures
\subsection{Structural Methods}
\subsubsection{Tamura's Texture Measures}

% \section{Classical Hough Transform}

% \section{Generalized Hough Transform}
% \subsection{R-Table Generation}

% \section{Representations of 2D Geometric Structures}
% \subsection{Boundary Representations}
% \subsubsection{Polyline}
% \subsubsection{Chain Code}
% \subsubsection{Curvature Scale Space (CSS)}

% \subsection{Region Representations}
% \subsubsection{Spatial Occupancy Array}
% \subsubsection{Axis-based Representations}
% \subsubsection{Quad-trees}
% % split and merge
% \subsubsection{Medial Axis Transform}

% % continue from here from "distances"

\end{multicols}

\end{document}
